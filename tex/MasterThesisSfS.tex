%%%--- Template for master thesis at SfS
%%%--- Modified template with more comments and examples -- SG, 11/06/09
%%%------
\documentclass[11pt,a4paper,twoside,openright]{report}
%%not needed \usepackage{E}
\usepackage[english]{style/ETHDAsfs}%--> ETHDASA + fancyheadings + ... "umlaute"
%  + sfs-hyper -> hyperref

\usepackage{pdfpages}%%to include the confirmation of originality (plagiarism
\usepackage{amsbsy}%% for \boldsymbol and \pmb{.}
\usepackage{amssymb}%% calls  amsfonts...
\usepackage{booktabs} % For \toprule etc. in tables
\usepackage[table]{xcolor} 
%or \usepackage{german8}%-- =  german  +  isolatin1
\usepackage{graphicx}%-- f?r PostScript-Grafiken (besser als  psfig!)
%\usepackage[draft]{graphicx} % grafics shown as boxes --> faster compilation
%
\usepackage[longnamesfirst]{natbib}%was {sfsbib}%- F?r  Literatur-Referenzen
%           ^^^^^^^^^^^^^^ 1) "Hampel, Ronchetti, ..,"  2) "Hampel et al"
% Engineers (and other funny people) want to see [1], [2]
% ---> use 'numbers' : \usepackage[longnamesfirst,number]{natbib}
%
%
\usepackage{style/texab}%- 'tex Abk?rzungen' /u/sfs/tex/tex/latex/texab.sty
        %%- z.B.  \R, \Z, \Q, \Nat f?r reelle, ganze, rationale, nat?rl. Zahlen;
        %%-       \N   (Normalvert.)  \W == Wahrscheinlichkeit .....
        %%-  \med, \var, \Cov, \....
        %%-  \abs{x} == |x|   und   \norm{y} ==  || y ||   (aber anst?ndig)
%% NOTE: texab contains many useful definitions and "shortcuts". It is
%% worth to open the file and have a look at them. HOWEVER, some
%% definitions can lead to conflicts with other packages. You
%% might for example want to comment out the line defininf \IF as an
%% operator when working with the algorithmic package, or to comment out
%% the line defining a command \Cite with working with the Biblatex package
\usepackage{amsmath}
%\usepackage{mathrsfs}% Raph Smith's Formal Script font --> provides \mathscr
\usepackage{enumerate}% Fuer selbstdefinierte Nummerierungen
%--------
\usepackage{relsize}%-> \smaller (etc) used here
\usepackage{color} %% to allow cloring in code listings
\usepackage{listings}% Fuer R-code, C-code, ....  and settings for these:
\definecolor{Mygrey}{gray}{0.75}% for linenumbers only!
\definecolor{Cgrey}{gray}{0.4}% for comments
\lstloadlanguages{R}
\lstset{ %% Hilfe unter z.B. http://en.wikibooks.org/wiki/LaTeX/Packages/Listings
language=R,
basicstyle=\ttfamily\scriptsize,%%- \small > \footnotesize > \scriptsize > \tiny
%commentstyle=\ttfamily\color{Cgrey},
commentstyle=\itshape\color{Cgrey},
numbers=left,
numberstyle=\ttfamily\color{Mygrey}\tiny,
stepnumber=1,
numbersep=5pt,
backgroundcolor=\color{white},
showspaces=false,
showstringspaces=false,
showtabs=false,
frame=single,
tabsize=2,
captionpos=b,
breaklines=true,
%breakatwhitespace=false,
keywordstyle={},
morekeywords={},
xleftmargin=4ex,
literate={<-}{{$$\leftarrow$$}}1 {~}{{$$\sim$$}}1}
\lstset{escapeinside={(*}{*)}} % for (*\ref{ }*) inside lstlistings (Scode)
%%----------------------------------------------------------------------------

%%------- Theoreme ---
\newtheorem{definition}{Definition}[subsection]
\newtheorem{lemma}[definition]{Lemma}
\newtheorem{theorem}[definition]{Theorem}
\newtheorem{Coro}[definition]{Corollary}
\theoremstyle{definition}
\newtheorem{example}[definition]{Example}
\newtheorem*{note}{Note}
\newtheorem*{remark}{Remark}

\DeclareMathOperator*{\plim}{plim}
% \def\MR#1{\href{http://www.ams.org/mathscinet-getitem?mr=#1}{MR#1}}

% \newcommand{\Lecture}[3]{\marginpar{#3.#2.#1}}
% \newcommand{\Fu}{\mathcal{F}}
\newcommand{\aatop}[2]{\genfrac{}{}{0pt}{}{#1}{#2}}

%\renewcommand{\theequation}{\arabic{equation}}
\numberwithin{equation}{subsection}

%%%%%%%%%%%%%%%%%%%%%%%%%%%%%%%%%%%%%%%%%%%%%%%%%
%%% Path for your figures                      %%%
%%%%%%%%%%%%%%%%%%%%%%%%%%%%%%%%%%%%%%%%%%%%%%%%%
% Set the paths where all figures are taken from:
\graphicspath{{images/}}

%%%%%%%%%%%%%%%%%%%%%%%%%%%%%%%%%%%%%%%%%%%%%%%%%
%%% Define your own commands here             %%%
%%%%%%%%%%%%%%%%%%%%%%%%%%%%%%%%%%%%%%%%%%%%%%%%%
\newcommand{\Bruch}[2]{{}^{#1}\!\!/\!_{#2}}
\renewcommand{\labelenumi}{\roman{enumi}.)}



\begin{document}
\bibliographystyle{style/chicago}% ---> Hampel,F., E.Ronchetti,... W.Stahel(1986) ...
 %was \bibliographystyle{sfsbib}\citationstyle{dcu} %OR DEFAULT : \citationstyle{agsm}

\pagenumbering{roman}%- roman numbering for first few pages

%%%%%%%%%%%%%%%%%%%%%%%%%%%%%%%%%%%%%%%%%%%%%%%%%
%%% Title page                                %%%
%%%%%%%%%%%%%%%%%%%%%%%%%%%%%%%%%%%%%%%%%%%%%%%%%
\period{$date$}
\dasatype{Master Thesis}
\students{$author$}
$if(coadviser)$
\alternatereaderprefix{Co-Adviser:}
\alternatereader{$coadviser$}
\mainreaderprefix{Adviser:}
\mainreader{$adviser$}
$else$
\alternatereaderprefix{Adviser:}
\alternatereader{$adviser$}
\mainreaderprefix{}
\mainreader{}
$endif$


\submissiondate{$submission_date$}
\title{$title$}

\maketitle%- Titelseite wird abgeschlossen
\cleardoublepage
 %%~~~~~~~~~~~~~~~~~~~~~~~~~~~~~~~~~~~~~~~~

%%%%%%%%%%%%%%%%%%%%%%%%%%%%%%%%%%%%%%%%%%%%%%%%%
%%% Insert here acknowledgements and abstract %%%
%%%%%%%%%%%%%%%%%%%%%%%%%%%%%%%%%%%%%%%%%%%%%%%%%
%% Dedication (optional)
\markright{}
\vspace*{\stretch{1}}
\begin{center}
    To some special person
\end{center}
\vspace*{\stretch{2}}

% Preface (optional)
\newpage
\markboth{Preface}{Preface}
\chapter*{Preface}

First words and acknowledgements.

%%% Local Variables: 
%%% mode: latex
%%% TeX-master: "MasterThesisSfS"
%%% End: 


% Abstract should not be longer than one page.
\newpage
\markboth{Abstract}{Abstract}
\chapter*{Abstract}

Short summary of my thesis.
 

%%% Local Variables: 
%%% mode: latex
%%% TeX-master: "MasterThesisSfS"
%%% End: 



%%%%%%%%%%%%%%%%%%%%%%%%%%%%%%%%%%%%%%%%%%%%%%%%%
%%% Table of contents and list of figures and %%%
%%% tables (no need to change this usually)   %%%
%%%%%%%%%%%%%%%%%%%%%%%%%%%%%%%%%%%%%%%%%%%%%%%%%
\newpage
\tableofcontents
\newpage
\listoffigures
\newpage
\listoftables

%% Notations and glossary (optional)
\cleardoublepage
\phantomsection
\addcontentsline{toc}{chapter}{\protect\numberline{}{Notation}}
\markboth{Notation}{Notation}
\chapter*{Notation}
\label{c:Notation}

Explain your symbols and abbreviations.


%%% Local Variables: 
%%% mode: latex
%%% TeX-master: "MasterThesisSfS"
%%% End: 


\cleardoublepage
\pagenumbering{arabic}%--- switch back to standard numbering


%%%%%%%%%%%%%%%%%%%%%%%%%%%%%%%%%%%%%%%%%%%%%%%%%
%%% Your text... Either write here directly,  %%%
%%% or even better: write in separate files   %%%
%%% that you just have to include here.       %%%
%%%%%%%%%%%%%%%%%%%%%%%%%%%%%%%%%%%%%%%%%%%%%%%%%
$body$
%%%%%%%%%%%%%%%%%%%%%%%%%%%%%%%%%%%%%%%%%%%%%%%%%
%%% Bibliography                              %%%
%%%%%%%%%%%%%%%%%%%%%%%%%%%%%%%%%%%%%%%%%%%%%%%%%
\addtocontents{toc}{\vspace{.5\baselineskip}}
\cleardoublepage
\phantomsection
\addcontentsline{toc}{chapter}{\protect\numberline{}{Bibliography}}
\bibliography{bib/bib}

%% All books from our library (SfS) are already in a BiBTeX file
%% (Assbib). You can use Assbib combined with your personal BiBTeX file:
%% \bibliography{Myreferences,Assbib}. Of course, this will only work on
%% the computers at SfS, unless you copy the Assbib file
%%  --> /u/sfs/bib/Assbib.bib

%%%%%%%%%%%%%%%%%%%%%%%%%%%%%%%%%%%%%%%%%%%%%%%%%
%%% Appendices (if needed)                    %%%
%%%%%%%%%%%%%%%%%%%%%%%%%%%%%%%%%%%%%%%%%%%%%%%%%
\addtocontents{toc}{\vspace{.5\baselineskip}}
\appendix
\chapter{Complementary information}
\label{app:complement}

Additional material. For example long mathematical derivations could be
given in the appendix. Or you could include part of your code that is
needed in printed form. You can add several Appendices to your thesis (as
you can include several chapters in the main part of your work).

\section{Including \Rp code with verbatim}
A simple (rather too simple, see~\ref{App:listings}) way to include code or
{\it R} output is to use 
\texttt{verbatim}. It just prints the text however it is (including all
spaces, ``strange'' symbols,...) in a slightly different font.
\begin{verbatim}
## loading packages
library(RBGL)
library(Rgraphviz)
library(boot)

## global variables
X_MAX <- 150

   This allows me to put as many s  p a   c es   as I want.
I can also use \ and ` and & and all the rest that is usually only 
accepted in the math mode.

I can also make as 
                  many 
             line 
    breaks as 
I want... and
             where I want. 
\end{verbatim}

\section{Including \Rp code with the \emph{listings} package}\label{App:listings}
However, it is much nicer to use the \emph{listings} package to include \Rp
code in your report. It allows you to number the lines, color the comments
differently than the code, and so on.

\lstinputlisting{images/picture.R}


\section{Using Sweave to include \Rp code (and more) in your report}
The easiest (and most elegant) way to include \Rp code and its output (and
have all your figures up to date with your report) is to use Sweave. You
can find an introduction Sweave in \texttt{/u/sfs/StatSoftDoc/Sweave/Sweave-tutorial.pdf}.

%%% Local Variables: 
%%% mode: latex
%%% TeX-master: "MasterThesisSfS"
%%% End: 

\chapter{Yet another appendix....}

\section{Description}
\begin{description}
\item[Something] details.
\item[Something else] other definition.
\end{description}

\section{Tables}
Refer to Table~\ref{tab:example} to see a left justified table with caption
on top.

\begin{table}[ht]
\centering
\caption[Test results]{\label{tab:example}Results.}
\begin{tabular}{ll}
\hline
\textbf{Student} & \textbf{Grade}\\
\hline
Marie  & $6$\\
Alain  & $5.5$\\
Josette  & $4.5$\\
Pierre  & $5$\\
\hline
\end{tabular}
\end{table}

%%% Local Variables: 
%%% mode: latex
%%% TeX-master: "MasterThesisSfS"
%%% End: 



%% Epilogue (optional)
\addtocontents{toc}{\vspace{.5\baselineskip}}
\cleardoublepage
\phantomsection
\addcontentsline{toc}{chapter}{\protect\numberline{}{Epilogue}}
\markboth{Epilogue}{Epilogue}
\chapter*{Epilogue}
\label{s:Epilogue}

A few final words.



%%% Local Variables: 
%%% mode: latex
%%% TeX-master: "MasterThesisSfS"
%%% End: 



%%%%%%%%%%%%%%%%%%%%%%%%%%%%%%%%%%%%%%%%%%%%%%%%%%
%%% Declaration of originality (Do not remove!)%%%
%%%%%%%%%%%%%%%%%%%%%%%%%%%%%%%%%%%%%%%%%%%%%%%%%%
%% Instructions:
%% -------------
%% fill in the empty document confirmation-originality.pdf electronically
%% print it out and sign it
%% scan it in again and save the scan in this directory with name
%% confirmation-originality-scan.pdf
%%
%% General info on plagiarism:
%% https://www.ethz.ch/students/en/studies/performance-assessments/plagiarism.html
\cleardoublepage
\includepdf[pages={-}, frame=true,scale=1]{pdf/confirmation-originality.pdf}
\end{document}
